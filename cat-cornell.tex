\documentclass{article}
\usepackage{calc}
\usepackage[top=0.25in,bottom=0.75in,left=0.5in,right=0.5in]{geometry}
\usepackage{tikz}
\usepackage{xeCJK}
 
\newlength{\wholeboxwd}
    \setlength{\wholeboxwd}{0.99\textwidth}
\newlength{\wholeboxht}
    \setlength{\wholeboxht}{0.95\textheight}
%
\newlength{\cuewd}
    \setlength{\cuewd}{0.3\wholeboxwd}
\newlength{\summht}
    \setlength{\summht}{0.2\wholeboxht}
\newlength{\cgridht}
    \setlength{\cgridht}{\wholeboxht-\summht}
\newlength{\cgridwd}
    \setlength{\cgridwd}{\wholeboxwd-\cuewd}
\newlength{\xorig}
\newlength{\yorig}
\setlength{\xorig}{0cm}
\setlength{\yorig}{0cm}

\pagestyle{empty}
% 删除页码
\begin{document}
 
\begin{center}
\begin{tikzpicture}
\draw[step=.5cm,gray!80,thin] (\cuewd,\summht) grid (\wholeboxwd,\wholeboxht);
%% Optional:
% \draw[step=.25cm,gray!30,thin] (\cuewd,\summht) grid (\wholeboxwd,\wholeboxht);
% Summary, top:
\draw [line width=.2pt] (\xorig,\summht) -- (\wholeboxwd,\summht);
% Grid, left:
\draw [line width=.2pt] (\cuewd,\summht) -- (\cuewd,\wholeboxht);
% Draw the big box:
\draw (\xorig,\yorig) rectangle (\wholeboxwd,\wholeboxht);
\node[anchor=west] at (0.25,\wholeboxht-1em) {\textbf{态射:}};
\node[anchor=west] at (0.25,\summht-1em){\textbf{函子:}};
\node[anchor=west,fill=white] at (\cuewd+1em,\wholeboxht-1em) {\textbf{\ 对象:\ }};
\node[anchor=west] at (\cuewd+1.4em,\wholeboxht+2em){%
    % \parbox[t]{\cuewd}
    \textbf{范畴:}%\par\smallskip 日期:
};

\node[anchor=west] at (\wholeboxwd-13.5em,\summht-13em){\textbf{一只寻找函子的猫的专属笔记}};
\end{tikzpicture}
\end{center}
\end{document}